% factorial.tex
\documentclass[12pt,letterpaper]{letter}
\usepackage{fullpage,color,graphicx}
\usepackage[procnames]{listings}
\pagenumbering{gobble}

\newcommand{\assignment}{Factorial}
\newcommand{\myinfo}
{Darin Critchlow
\\CSIS 2430-001}
\newcommand{\assignmentDescription}
{With your newly chosen programming language, your goal is to implement factorial for a prompted integer. Make sure you include in your analysis how well the algorithm worked with your newly chosen language. When does it fail? Why?}
\newcommand{\worked}
{The algorithm worked well in python. It was very easy to grab the user input from the console. Python is built very nicely to run scripts like this factorial script.}
\newcommand{\didntwork}
{I had to research factorials of 0 and 1 so that the program would return the correct answer. I had forgotten that both 0 and 1 return 1.}
\newcommand{\comments}
{I am very happy so far with the python language and I am excited to program some web applications with it.}

\begin{document}

\LARGE\textbf{\assignment{}}

\normalsize\myinfo{}

\textbf{\underline{Objective:}}

\textbf{\assignmentDescription{}}

\textbf{\underline{What Worked:}}

\worked{}

\textbf{\underline{What Didn't Work:}}

\didntwork{}

%\pagebreak
\textbf{\underline{Comments:}}

\comments{}

\pagebreak
\definecolor{keywords}{RGB}{255,127,0}
\definecolor{comments}{RGB}{169,169,169}
\definecolor{red}{RGB}{160,0,0}
\definecolor{green}{RGB}{46,139,87}

\lstset{language=Python,
        basicstyle=\ttfamily\small,
        keywordstyle=\color{keywords},
        commentstyle=\color{comments},
        stringstyle=\color{green},
        showstringspaces=false,
        identifierstyle=\color{black},
        procnamekeys={def,class}}
\textbf{\underline{Code:}}
\lstinputlisting[language=Python]{factorial.py}

\includegraphics[width=470pt]{factorial}
\end{document}
